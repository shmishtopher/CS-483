%
% @author   Shmish  "shmish90@gmail.com"
% @legal    MIT     "(c) Christopher Schmitt"
%


\documentclass{article}


%
% Document Imports
%

\usepackage{fancyhdr}
\usepackage{extramarks}
\usepackage{amsmath}
\usepackage{amssymb}
\usepackage{amsthm}
\usepackage{amsfonts}
\usepackage{color}
\usepackage{tikz}
\usepackage{listings}



%
% Document Configuration
%

\newcommand{\hwAuthor}{Christopher K. Schmitt}
\newcommand{\hwSubject}{CS 483}
\newcommand{\hwSection}{Section 01}
\newcommand{\hwSemester}{Fall 2020}
\newcommand{\hwAssignment}{Assignment 5}

\usetikzlibrary{arrows,automata}
\usetikzlibrary{calc}


%
% Document Environments
%

\setlength{\headheight}{65pt}
\pagestyle{fancy}
\lhead{\hwAuthor}
\rhead{
  \hwSubject \\
  \hwSection \\
  \hwSemester \\
  \hwAssignment
}

\newenvironment{problem}[1]{
  \nobreak\section*{Problem #1}
}{}


%
% Document Start
%

\begin{document}
  \begin{problem}{1}
    \begin{enumerate}
      \item[a.] $f$ is not one-to-one, both $1$ and $3$ map to $6$
      \item[b.] $f$ is not onto, $f$ does not cover $10$
      \item[c.] $f$ is a  correspondence 
      \item[d.] $g$ is one-to-one
      \item[e.] $g$ is onto
      \item[f.] $g$ is a  correspondence    
    \end{enumerate}
  \end{problem}

  \begin{problem}{2}
    \begin{proof}
      $ $\\
      Suppose $\mathcal{B}$ is countable.  Then there is a
      correspondence between the naturals, $\mathbb{N}$, and
      $\mathcal{B}$.  We define a function, $f(n)$, which maps each
      natural to an element in $\mathcal{B}$.  It may look something
      like this:
      
      \begin{center}
        \begin{displaymath}
          \begin{tabular}{ll}
            $n$ & $f(n)$ \\ \hline
            $1$ & $01000101\dots$ \\
            $2$ & $10011101\dots$ \\
            $3$ & $01111101\dots$ \\
            $4$ & $01010000\dots$ \\
            $\vdots$ & $\vdots$
          \end{tabular}
        \end{displaymath}
      \end{center}

      \noindent
      now suppose we take the sequence $s \in \mathcal{B}$, which is
      constructed like so: $s_i = 0$ when $f(i)_i = 1$, and $s_i = 1$
      when $f(i)_i = 0$.  Essentially, the $i$'th position of $s$ is
      always the opposite of a single bit in every $f(n)$.  Because
      $s \in \mathcal{B}$, but $f$ cannot produce $s$, there is no
      bijection between $\mathcal{B}$ and $\mathbb{N}$, so
      $\mathcal{B}$ is not countable.
      $ $\\
    \end{proof}
  \end{problem}

  \begin{problem}{3}
    \begin{proof}
      $ $\\
      Suppose $T$ is decided by $\mathcal{R}$.  We can construct TM
      $\mathcal{Z}$ as follows:

      \begin{center}
        $\mathcal{Z}$, on input $<M, w>$: \begin{enumerate}
          \item Create a TM $\mathcal{Z}_0$ as follows.  On input $x$: \begin{enumerate}
            \item Reject if $x$ is not 10, or 01
            \item Accept if $x$ is 01
            \item Run $M$ on $w$, and accept if $M$ accepts 
          \end{enumerate}
          \item Run $\mathcal{R}$ on $<\mathcal{Z}_0>$
          \item Accept if $\mathcal{R}$ accepts, otherwise reject
        \end{enumerate}
      \end{center}

      \noindent
      We know that $T$ in undecidable because $\mathcal{Z}$ decides
      $A_{TM}$ (which is not decidable).
      $ $\\
    \end{proof}
  \end{problem}
\end{document}