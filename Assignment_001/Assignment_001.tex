%
% @author   Shmish  "shmish90@gmail.com"
% @legal    MIT     "(c) Christopher Schmitt"
%


\documentclass{article}


%
% Document Imports
%

\usepackage{fancyhdr}
\usepackage{extramarks}
\usepackage{amsmath}
\usepackage{amssymb}
\usepackage{amsthm}
\usepackage{amsfonts}
\usepackage{color}
\usepackage{tikz}



%
% Document Configuration
%

\newcommand{\hwAuthor}{Christopher K. Schmitt}
\newcommand{\hwSubject}{CS 483}
\newcommand{\hwSection}{Section 01}
\newcommand{\hwSemester}{Fall 2020}
\newcommand{\hwAssignment}{Assignment 1}


%
% Document Environments
%

\setlength{\headheight}{65pt}
\pagestyle{fancy}
\lhead{\hwAuthor}
\rhead{
  \hwSubject \\
  \hwSection \\
  \hwSemester \\
  \hwAssignment
}

\newenvironment{problem}[1]{
  \nobreak\section*{Problem #1}
}{}


%
% Document Start
%

\begin{document}
  \begin{problem}{1}
    Let $A$ be the set $\{x, y, z\}$ and $B$ be the set $\{x, y\}$

    \begin{enumerate}
      \item $A$ is not a subset of $B$
      \item $B$ is a proper subset of $A$
      \item $A \cup B = \{x, y, z\}$
      \item $A \cap B = \{x, y\}$
      \item $A \times B = \{(x, x), (x, y), (y, x), (y, y), (z, x), (z, y)\}$
      \item $\mathcal{P}(B) = \{\emptyset, \{x\}, \{y\}, B\}$
    \end{enumerate}
  \end{problem}

  \begin{problem}{2}
    If $A$ has $a$ elements, and $B$ has $b$ elements, how many 
    elements are in $A \times B$?

    \begin{center}
      $|A \times B| = ab$
    \end{center}

    \noindent
    This is because, in order to pair every member of $A$ with
    each memeber of $B$, $|A|\cdot|B|$ tuples are required.
  \end{problem}

  \begin{problem}{3}
    If $C$ is a set with c elements, how many elements are in 
    the power set of C?

    \begin{center}
      $|\mathcal{P}(C)| = 2^{c}$
    \end{center}

    \noindent
    Each element of of $\mathcal{P}(C)$ can either contain or
    exclude every element of $C$.  There will always be exactly
    $2^{c}$ unique ways to do this.
  \end{problem}

  \begin{problem}{4}
    Let $X$ be the set $\{1, 2, 3, 4, 5\}$ and $Y$ be the set
    $\{6, 7, 8, 9, 10\}$.  Let $f : X \to Y$ and $g : X \times Y \to Y$

    \begin{enumerate}
      \item $f(2) = 7$
      \item The range of $f$ is $Y$, the domain of $f$ is $X$
      \item $g(2, 10) = 6$
      \item The range of $g$ is $Y$, the domain of $g$ is $X \times Y$
      \item $g(4, f(4)) = 8$
    \end{enumerate}
  \end{problem}

  \begin{problem}{5}
    \begin{enumerate}
      \item $R = \{(a, a), (b, b), (c, c), (a, b), (b, a), (b, c), (c, b)\}$
      \item $R = \{(a, a), (b, b), (c, c), (a, b), (a, c), (b, c)\}$
      \item $R = \{(a, b), (b, a), (a, c), (c, a), (b, c), (c, b)\}$
    \end{enumerate}
  \end{problem}

  \begin{problem}{6}
    \begin{center}
      \begin{tikzpicture}
        \node[circle, draw] (1) at (0, 0) {$1$};
        \node[circle, draw] (2) at (3, 0) {$2$};
        \node[circle, draw] (3) at (3, 3) {$3$};
        \node[circle, draw] (4) at (0, 3) {$4$};

        \draw (1) -- (2);
        \draw (2) -- (3);
        \draw (2) -- (4);

        \draw[red, thick] (1) -- (3);
        \draw[red, thick] (1) -- (4);
      \end{tikzpicture}
    \end{center}

    \begin{table}[h]
      \centering
      \begin{tabular}{ll}
      Node & Degree \\ \hline
      1 & 3 \\
      2 & 3 \\
      3 & 2 \\
      4 & 2
      \end{tabular}
    \end{table}
  \end{problem}

  \begin{problem}{7}
    \begin{equation*}
      \begin{gathered}
        G = (V, E)\\
        V = \{1, 2, 3, 4, 5, 6\}\\
        E = \{\{1, 4\}, \{1, 5\}, \{1, 6\}, \{2, 4\}, \{2, 5\}, \{2, 6\}, \{3, 4\}, \{3, 5\}, \{3, 6\}\}
      \end{gathered}
    \end{equation*}
  \end{problem}

  \begin{problem}{8}
    If $a = b$, then $a - b = 0$.  The error in the proof is the
    division by $(a - b)$.  This operation is undefined when the
    denominator is $0$, therefore the proof is invalid.
  \end{problem}

  \begin{problem}{9}
    \theoremstyle{definition}
    \newtheorem*{theorem}{Theorem}

    \begin{theorem}
      $S(n) = 1 + 2 + 3 + \dotsb + n = \frac{n(n+1)}{2}$
    \end{theorem}

    \begin{proof}
      By induction on $n$

      \begin{description}
        \item[Base:] When $n = 1$, $S(n) = 1 = \frac{1(1 + 1)}{2}$
        \item[Inductive:] Suppose $S(k) = \frac{k(k + 1)}{2}$\\
        $\implies S(k + 1) = 1 + 2 + 3 + \dotsb + k + (k + 1)$\\
        $\implies S(k + 1) = S(k) + (k + 1) = \frac{k(k + 1)}{2} + (k + 1)$\\
        $\implies S(k + 1) = \frac{(k + 1)((k + 1) + 1)}{2} = 1 + 2 + \dotsb + k + (k + 1)$  
      \end{description}
    \end{proof}
  \end{problem}

  \begin{problem}{10}
    \begin{theorem}
      For any $n \in \mathbb{Z}$, if $n^3+5$ is odd then $n$ is even.
    \end{theorem}

    \begin{proof}
      Suppose that, if $n^3+5$ is odd then $n$ is also odd.
      \begin{equation*}
        \begin{split}
          \implies & n^3 \text{ is odd, because the product of odd numbers is odd}\\
          \implies & n^3 + 5 \text{ is even, because the sum of two odd numbers is even}\\
          \therefore{} & n^3 + 5 \text{ is even, and then supposition is incorrect}
        \end{split}
      \end{equation*}
    \end{proof}
  \end{problem}

  \begin{problem}{11}
    \begin{theorem}
      In a set of $51$ random integers in $[1, 100]$, there are at least two integers
      that divide eachother without remainder.
    \end{theorem}

    \begin{proof}
      Partition the set of $51$ integers such that each subset conforms to
      the relation that each element of the subset is a multiple of another
      element of the subset.  If this is down by grouping multiples of odd
      numbers such that: $\{k, 2k, 4k, 8k, \dotsc , 2^ik\}$, where $k$ is
      any odd number in $[1, 100]$, we will have 50 subsets.  By the 
      pigeonhole principle, $\lceil \frac{51}{50} \rceil = 2$, so at least
      $2$ random elements will be part of the same subset.  Because subsets
      are constructed by their multiples, the larger one will divide by the 
      smaller one without remainder.
    \end{proof}
  \end{problem}
\end{document}